\documentclass[%
	11pt,
	a4paper,
	utf8,
	%twocolumn
		]{article}	

\usepackage{style_packages/podvoyskiy_article_extended}


\begin{document}
\title{Машинное обучение и анализ данных. Заметки}

\author{\itshape Подвойский А.О.}

\date{}
\maketitle

\thispagestyle{fancy}

\tableofcontents

\section{Калибровка классификаторов}

Подробности в статье А. Дьяконова \href{https://dyakonov.org/2020/03/27/%D0%BF%D1%80%D0%BE%D0%B1%D0%BB%D0%B5%D0%BC%D0%B0-%D0%BA%D0%B0%D0%BB%D0%B8%D0%B1%D1%80%D0%BE%D0%B2%D0%BA%D0%B8-%D1%83%D0%B2%D0%B5%D1%80%D0%B5%D0%BD%D0%BD%D0%BE%D1%81%D1%82%D0%B8/}{<<Проблема калибровки уверенности>>}.
\vspace{2mm}

Ниже описываются способы оценить качество калибровки алгоритма. Надо сравнить \emph{уверенность} (confidence) и \emph{долю верных ответов} (accuracy) на тестовой выборке.

Если классификатор <<хорошо откалиброван>> и для большой группы объектов этот классификатор возвращает вероятность принадлежности к положительному классу 0.8, то среди этих объектов будет приблизительно 80\% объектов, которые в действительности принадлежат положительному классу. То есть, если для группы точек данных общим числом 100 классификатор возвращает вероятность положительного класса 0.8, то приблизительно 80 точек на самом деле будут принадлежать положительному классу и доля верных ответов тогда составит 0.8.

\subsection{Непараметрический метод гистограммной калибровки (Histogram Binning)}

Изначально в методе использовались бины одинаковой ширины, но можно использовать и равномощные бины.

Недостатки подхода:

\begin{itemize}
	\item число бинов задается наперед,
	
	\item функция деформации не непрерывна,
	
	\item в <<равноширинном варианте>> в некоторых бинах может содержаться недостаточное число точек.
\end{itemize}

Метод был предложен Zadrozny В. и Elkan C. \href{http://cseweb.ucsd.edu/~elkan/calibrated.pdf}{\ttfamily Obtaining  calibrated  probability  estimates  from  decision  trees  and naive bayesian classifiers}.

\subsection{Непараметрический метод изотонической регрессии (Isotonic Regression)}

Строится монотонно неубывающая функция деформации оценок алгоритма.

Метод был предложен Zadrozny B. и Elkan C. \href{http://citeseerx.ist.psu.edu/viewdoc/download?doi=10.1.1.13.7457&rep=rep1&type=pdf}{\ttfamily Transforming classifier scores into accurate multiclass probability estimates}.

Функция деформации по-прежнему не является непрерывной.

\subsection{Параметрическая калибровка Платта (Platt calibration)}

Изначально этот метод калибровки разрабатывался только для метода опорных векторов, оценки которого лежат на вещественной оси (по сути, это расстояния до оптимальной разделяющей классы прямой, взятые с нужным знаком). Считается, что этот метод не очень подходит для других моделей.

Предложен Platt~J. \href{http://citeseerx.ist.psu.edu/viewdoc/download;jsessionid=EA4888FEE74FB677B492740F59CDFE1F?doi=10.1.1.41.1639&rep=rep1&type=pdf}{\ttfamily Probabilistic  outputs  for  support  vector machines and comparisons to regularized likelihood methods}.

\subsection{Логистическая регрессия в пространстве логитов}

\subsection{Деревья калибровки}

Стандартный алгоритм строит строит суперпозицию дерева решений на исходных признаках и логистических регрессий (каждая в своем листе) над оценками алгоритма:

\begin{itemize}
	\item Построить на исходных признаках решающее дерево (не очень глубокое),
	
	\item В каждом листе -- обучить логистическую регрессию на одном признаке,
	
	\item Подрезать дерево, минимизируя ошибку.
\end{itemize}

\subsection{Температурное шкалирование (Temperature Scaling)}

Этот метод относится к классу DL-методов калибровки, так как он был разработан именно для калибровки нейронных сетей. Метод представляет собой простое многомерное обобщение шкалирования Платта.




% Источники в "Газовой промышленности" нумеруются по мере упоминания 
\begin{thebibliography}{99}\addcontentsline{toc}{section}{Список литературы}
	\bibitem{chacon:2020}{ \emph{Чакон С.}, \emph{Штрауб Б.} Git для профессионального программиста. -- СПб.: Питер, 2020. -- 496~с. }
	
\end{thebibliography}

%\listoffigures\addcontentsline{toc}{section}{Список иллюстраций}

\end{document}
